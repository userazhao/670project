\section{Datasets}
\label{sec:datasets}

Given that this project does not use any neural networks, we do not require a very large dataset to train any of our algorithms. We will however require a testing dataset to test the runtime and the qualitative outputs of our algorithms. For this purpose, we believe that it would be best to take multiple pictures of a textured scene with a changing foreground in succession to be used as inputs and reference images. For example, different images each containing a different object against an area of wooden floor. Then, the input image can be one image with the object cut out, while the reference images can be the other images. It's important that the uncut input image is not included as a reference image, because then the algorithms are likely to reproduce that object rather than an empty wooden floor.

To test a more probable real life situation, we can also record a short video and use the frames of that video as inputs and references.

It is best to create the datasets for this project so that there will be no copyright issues and pictures can be chosen which best exhibit the differences and capabilities of our algorithms. Additionally, the pictures or videos will be compressed or cropped as necessary to demonstrate the algorithms within a reasonable runtime.