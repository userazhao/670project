\begin{introduction}

\section{Problem Definition}

In the vast realm of computer vision and image processing, a particularly daunting challenge lies in the domain of texture synthesis, specifically when it comes to filling in the gaps within images. This process, often referred to as hole filling, demands the ability to infer what should occupy these void spaces while ensuring that the surrounding visual elements maintain their coherence and structure. However, the conventional methods we rely on for such tasks often stumble, particularly when faced with intricate textures or sizable gaps.



Motivation: 
The drive behind this project stems from the recognition of the limitations inherent in current texture synthesis methods and the promise of enhancing both efficiency and effectiveness. While methods like the non-parametric sampling technique pioneered by Efros and Leung and the PatchMatch algorithm devised by Barnes et al. have shown potential, they still grapple with certain challenges. These include struggles with maintaining overall coherence or efficiently handling larger and more complex textures, especially when structural hole filling or multiple reference images are involved.

Furthermore, there exists a compelling opportunity to harness recent advancements in fields like machine learning, optimization techniques, and multi-reference synthesis. By doing so, we can craft algorithms that are not only more adept at texture synthesis and hole filling but also capable of unlocking new applications in areas like image restoration, inpainting, and augmented reality, where preserving visual fidelity is paramount.

Overview: Within the confines of this project, our primary aim is to delve deep into the existing landscape of texture synthesis techniques. Drawing inspiration from a diverse array of sources, including academic literature and real-world implementation, we endeavor to pioneer novel algorithms that excel in robust texture synthesis and structural hole filling. By leveraging insights gleaned from seminal works such as "Texture synthesis by non-parametric sampling" by Efros and Leung and "PatchMatch: A randomized correspondence algorithm for structural image editing" by Barnes et al., among others, we hope to chart new territories in this domain.

Moreover, we are keen on exploring methodologies that seamlessly integrate multiple reference images, thereby enriching the synthesis process and bolstering its versatility. Through rigorous empirical evaluation and rigorous comparison with the current state-of-the-art methods, we aim to showcase the tangible benefits and efficacy of our proposed approaches. Ultimately, our endeavors are driven by a collective aspiration to overcome the formidable challenges posed by texture synthesis and pave the way for significant advancements in the realms of computer vision and image processing.


\end{introduction}
