\begin{introduction}

\section{Problem Definition}

Texture synthesis, particularly in the context of hole filling within images, poses a significant challenge in computer vision and image processing. The task involves inferring missing or damaged regions of an image while preserving the visual coherence and structure of the surrounding content. Traditional methods often struggle with maintaining consistency and realism, especially when dealing with complex textures or large holes.



Motivation: In the realm of computer vision and image processing, the task of texture synthesis and hole filling presents a significant challenge. Traditional methods often struggle to effectively fill in missing or damaged regions within images while maintaining visual coherence and realism. This problem becomes particularly complex when dealing with intricate textures or large holes.

Motivated by the shortcomings of existing techniques, this project aims to explore innovative approaches to texture synthesis and hole filling. By leveraging advancements in machine learning, optimization algorithms, and multi-reference synthesis, we seek to develop more efficient and effective solutions. The ultimate goal is to address the limitations of current methods and contribute to advancements in image restoration, inpainting, and augmented reality applications.

\end{introduction}
