\begin{methods}
\section{Methods}
\subsection{Overview of proposed methods}

We implemented the texture synthesis algorithm based on Efros and Leung's non-parametric sampling method. It includes optimizations for hole filling and integration with multiple reference images. Our script implements the PatchMatch algorithm by Barnes et al., which generates textures by copying patches from the input. We have extended this algorithm to support structural hole filling and integration with multiple reference images.
Additional Enhancements: We have developed additional enhancements and optimizations, including improved search strategies, adaptive patch selection, and parallel processing, to further enhance the performance and effectiveness of our method.

 By focusing on structural hole filling and integrating multiple reference images, our method is expected to produce results with higher structural coherence and realism compared to existing approaches. The integration of multiple texture synthesis algorithms and the ability to work with diverse input data make our method more versatile and adaptable to a wide range of image inpainting scenarios. Through optimization techniques and algorithmic improvements, we aim to achieve superior runtime performance, making our method suitable for both offline and real-time applications.

\subsection{Intuition on Why Our Method Could be Better Than the State of the Art}

By integrating various texture synthesis algorithms such as Efros and Leung's method, PatchMatch, and possibly others, our method benefits from a diverse range of techniques. This integration allows us to leverage the strengths of each algorithm and mitigate their individual limitations, resulting in more robust and adaptable hole filling. Unlike some existing approaches that prioritize textural consistency over structural coherence, our method emphasizes both aspects equally. This focus on structural coherence ensures that the synthesized textures seamlessly blend with the surrounding context, leading to more visually convincing results, especially in scenarios with complex structures or irregular hole shapes.
Our method prioritizes optimization techniques and algorithmic enhancements to achieve superior runtime performance. By leveraging parallel processing, adaptive search strategies, and memory management optimizations, we aim to make our method efficient and scalable, suitable for both offline processing and real-time applications.The incorporation of multiple reference images allows our method to access a richer source of information during the synthesis process. This enables better adaptation to diverse textures and enhances the overall quality and realism of the synthesized results, especially in scenarios where a single reference image may not provide sufficient context.

\subsection{Details of Models and Algorithms Developed}

SynthEfrosLeung.py: This Python script implements the texture synthesis algorithm based on Efros and Leung's non-parametric sampling method. It includes optimizations for hole filling and integration with multiple reference images. The algorithm involves iteratively copying pixels from the input images based on a minimal distance metric, with additional enhancements for handling hole regions.
PatchMatch.py: This script implements the PatchMatch algorithm by Barnes et al., which generates textures by copying patches from the input. We have extended this algorithm to support structural hole filling and integration with multiple reference images. The algorithm iteratively refines a nearest neighbor field (NNF) by propagating better matches from nearby patches and performing random searches based on an exponentially shrinking search radius.

We have developed additional enhancements and optimizations, including improved search strategies, adaptive patch selection, and parallel processing, to further improve the performance and effectiveness of our method. These enhancements aim to address specific challenges in texture synthesis and hole filling, such as handling irregular hole shapes and preserving structural coherence. We performed the qualitative evaluations by visually inspecting the synthesized results and comparing them to state-of-the-art techniques. This visual assessment will involve evaluating factors such as structural coherence, textural consistency, and overall realism to determine the subjective quality of the synthesized textures.We analyzed the robustness of our method to various types of input data, including images with different textures, sizes, and levels of damage or missing regions. This analysis will help identify potential limitations and areas for further improvement, ensuring the reliability and applicability of our method across diverse scenarios.

By conducting comprehensive experiments and evaluations, our model validates the effectiveness, efficiency, and robustness of our proposed method for texture synthesis and hole filling, demonstrating its superiority over existing state-of-the-art techniques. This elaboration provides a detailed insight into the intuition behind our proposed method's potential superiority, the models and algorithms developed, and the experiments planned for validation and evaluation.


\end{methods}