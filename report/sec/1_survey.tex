\section{Survey}
\label{sec:survey}

A brief survey of papers related to the project. Some of these papers describe algorithms which will be directly used in the project, while others describe subjects which are related. All papers are related to textural hole filling.

%-------------------------------------------------------------------------
\subsection*{Texture synthesis by non-parametric sampling \cite{efros1999texture}}
This paper by Efros and Leung describes the non-parametric sampling method for texture synthesis that was used in Homework 2. The algorithm given in the paper involves first copying a $3\times3$ pixel area into the center of an empty image, and then growing the image by copying pixels out from the center by copying pixels from the input based on a minimal distance metric. The paper mentions hole filling, but the implementation specifics for turning the texture synthesis algorithm into a hole filling algorithm are not given. However, it should be relatively easy to do, instead of initializing the texture by generating a $3\times3$ pixel region in the center, we will just build off the existing input image. Additionally, we can add the ability to draw pixels from multiple reference images by simply treating those images as input images which we will read pixels from but not write to. One thing to note is that this method, when applied to hole filling, is considered a textural hole filling method rather than a structural one. This means that it will work well for textures, like the pattern of a wall, but may not work well with structures, like a flagpole.
\subsection*{PatchMatch: A randomized correspondence algorithm for structural image editing \cite{barnes2009patchmatch}}
This paper by Barnes \etal describes the PatchMatch algorithm which generates textures by copying patches from the input. The first step of the algorithm is to initialize a nearest neighbor field (NNF), which contains the distance between each patch of the input and the nearest neighbor of the output. The NNF is then iteratively improved by first propagating better matches from nearby patches and then performing a random search of patches based on an exponentially shrinking search radius.
Like the Efros Leung method, PatchMatch is also non-parametric, however, it samples patches rather than pixels and does not recompute the ``distance" metric for each patch/pixel, saving on runtime. PatchMatch has many different applications, and can perform both textural and structural hole filling. For this project, I plan to only implement textural hole filling. While the paper mentions hole filling as a possibility, the exact algorithm for it is not described, but it should be easy to go from the texture synthesis algorithm to a hole filling algorithm.

The paper claims that the runtime of the PatchMatch algorithm is several orders of magnitude faster than the Efros Leung algorithm, and since a faster runtime is one of the goals of the project, this is promising.
\subsection*{Synthesizing natural textures \cite{ashikhmin2001synthesizing}}
This paper by Ashikhmin describes an algorithm which outspeeds the Efros Leung algorithm by encouraging copying of larger patches from the input image. This algorithm is similar to the quilting algorithm described in Homework 2, however the patches are irregularly shaped, which makes it better suited for natural textures. This is one of the papers cited by the PatchMatch paper, and we may implement this algorithm as a third comparison depending on available time.
\subsection*{Region filling and object removal by exemplar-based image inpainting \cite{criminisi2004region}}
This paper by Criminisi \etal describes an algorithm which works for filling both small and large holes in images. The algorithm works by filling patches of the hole in order of highest priority, determined by the level of confidence in filling the patch, to avoid artifacting. Since the other three papers mentioned do not specifically address hole filling, this paper may be useful in modifying the Efros Leung and PatchMatch algorithms into hole filling algorithms.
