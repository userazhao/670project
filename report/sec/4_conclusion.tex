\begin{conclusion}
    
    
\section{Conclusion}
\subsection{Discussion and future work}

In conclusion, our proposed method offers a promising solution for texture synthesis and hole filling, with competitive performance, efficiency, and robustness compared to state-of-the-art techniques. The experiments demonstrate the effectiveness of our method in addressing common challenges in image inpainting and pave the way for future research directions, including:

Further optimization and refinement of algorithms to enhance performance and scalability. Exploration of advanced machine learning techniques for improved texture synthesis and inpainting. Integration of user feedback mechanisms for interactive inpainting applications. Extension of the method to support 3D texture synthesis and inpainting for applications in computer graphics and augmented reality. Overall, our study contributes to the advancement of texture synthesis and hole filling techniques, with implications for various domains such as image editing, restoration, and content creation.

This comprehensive overview provides detailed insights into the experimental setup, the questions addressed by the experiments, the results obtained, and the implications for future research and development.



\end{conclusion}