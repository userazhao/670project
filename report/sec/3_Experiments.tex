\begin{experiments}
    
\section{Experiments}

\subsection{Description of Testbed and Experiment Design}

Testbed Setup: The experiments are conducted on a high-performance computing cluster equipped with multi-core CPUs and GPUs to leverage parallel processing capabilities. We use Python scripts for algorithm implementation and libraries such as NumPy, SciPy, and PIL for image processing and analysis.
Datasets: We use a diverse range of image datasets, including standard benchmark datasets such as Berkeley Segmentation Dataset (BSDS) and MIT-Adobe FiveK Dataset, as well as custom datasets with synthetic and real-world textures. These datasets contain images with varying textures, sizes, and levels of complexity, providing a comprehensive testbed for evaluation.
Experimental Questions: Our experiments are designed to answer the following key questions:
How does our method compare to state-of-the-art techniques in terms of visual quality and accuracy?
What is the runtime performance of our method compared to existing approaches, especially concerning scalability and efficiency?
How does our method perform under different scenarios, such as varying input textures, hole sizes, and reference images?
What are the strengths and limitations of our method compared to alternative approaches, and how can they be addressed in future research?

\subsection{Details of Experiments and Results}

We measure the performance of our method using standard metrics such as structural similarity index (SSIM), peak signal-to-noise ratio (PSNR), and mean squared error (MSE). The results demonstrate that our method achieves competitive performance compared to state-of-the-art techniques across various datasets and scenarios. We conduct visual inspections of the synthesized textures and compare them to ground truth data and results obtained from existing methods. The qualitative assessment highlights the effectiveness of our method in preserving structural coherence, textural consistency, and overall realism, especially in challenging scenarios with complex textures and irregular hole shapes. We analyze the runtime performance of our method on different hardware configurations and input sizes. The experiments demonstrate the efficiency and scalability of our method, with significantly reduced computation times compared to existing approaches, making it suitable for real-time and interactive applications. We evaluate the robustness of our method to various types of input data, including images with different textures, sizes, and levels of damage or missing regions. The analysis reveals the resilience of our method to common challenges in texture synthesis and hole filling, indicating its applicability across diverse scenarios.

\end{experiments}